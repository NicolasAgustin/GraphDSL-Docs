\documentclass{article}
\title{GraphDSL Documentacion}
\author{Nicolas Sandez}
\usepackage{amsmath}
\usepackage{xcolor}
\usepackage{listings}
\begin{document}

    \definecolor{dkgreen}{rgb}{0,0.6,0}
    \definecolor{gray}{rgb}{0.5,0.5,0.5}
    \definecolor{mauve}{rgb}{0.58,0,0.82}
    \definecolor{backcolour}{rgb}{0.95,0.95,0.92}

    \lstset{
            backgroundcolor=\color{backcolour},
            frame=tb,
            language=Python,
            aboveskip=3mm,
            belowskip=3mm,
            showstringspaces=false,
            columns=flexible,
            basicstyle={\small\ttfamily},
            numbers=none,
            numberstyle=\tiny\color{gray},
            keywordstyle=\color{blue},
            commentstyle=\color{dkgreen},
            stringstyle=\color{mauve},
            breaklines=true,
            breakatwhitespace=true,
            tabsize=3
    }

    \lstset{language=Python}

	\pagenumbering{arabic}
	
	\maketitle
	\newpage
	
	
	\tableofcontents
	\newpage
	
	% \begin{equation*}
	% 	f(x) = x^2
	% \end{equation*}
	\section{Propuesta}
	El proyecto consiste en un lenguaje especifico de dominio para la generación de diagramas de
grafos en LaTeX. Dicho lenguaje, que lo llamaremos GraphDSL, cuenta con bucles, salida por
pantalla, condicionales, variables enteras y cadenas de caracteres.
	\section{Propiedades}
	El lenguaje soporta los tipos de datos: Enteros y Strings. A continuacion se detallan las operaciones disponibles para cada uno.
	\subsection{Enteros}
	\begin{itemize}
		\item Suma
		\item Resta
		\item División
		\item Multiplicación
		\item Igual que
		\item Menor
		\item Mayor
		\item Menor o igual
		\item Mayor o igual
		\item Distinto
		\item Casteo a string
	\end{itemize}
	\subsection{String}
	\begin{itemize}
		\item Concatenación
	    \item Longitud
	    \item Casteo a entero
	\end{itemize}
    \subsection{Nodos}
    Si bien no es posible definir un tipo de dato nodo por parte del usuario, 
    internamente son representados por un tipo de dato Nodo, el cual tiene las siguientes operaciones:
    \begin{itemize}
       \item{$\rightarrow$ Arista dirigida hacia la derecha.}
       \item{$\leftarrow$ Arista dirigida hacia la izquierda.}
       \item{$\leftrightarrow$ Arista bidireccional.}
       \item Color de nodo.
    \end{itemize}
    \subsection{Implementacion}
    Internamente el lenguaje opera como una matriz, donde cada nodo estara posicionado en una coordenada especifica.
    El tamaño de la matriz sera definido por el usuario y cada vez que se desee insertar un nodo a al grafo se debe
    especificar en que fila y columna colocarlo.
    \subsection{Colores}
    Tanto los nodos como las aristas permiten establecer el color de las mismas. Los colores disponibles son:
    \begin{itemize}
        \color{red}
        \item red
        \color{blue}
        \item blue
        \color{green}
        \item green
        \color{yellow}
        \item yellow
        \color{black}
        \item black
        \color{black}
        \item white
        \color{brown}
        \item brown
        \color{purple}
        \item purple
        \color{gray}
        \item grey
        \color{orange}
        \item orange
        \color{pink}
        \item pink
    \end{itemize}

    \newpage
    \section{Funciones nativas}
    \subsection{len}
    La funcion len devuelve la longitud de una cadena de caracteres.
    \begin{align} 
        \textbf{len}( \textbf{strexp} ) \rightarrow \textbf{int}
    \end{align}
    \begin{lstlisting}
        # Longitud de una cadena
        string cadena = "ejemplo";
        int longitud = len(cadena)
    \end{lstlisting}
    \subsection{int}
    La funcion int devuelve la representacion numerica de una cadena de caracteres.
    \begin{align} 
        \textbf{int}( \textbf{strexp} ) \rightarrow \textbf{int}
    \end{align}
    \begin{lstlisting}
        # Casteo de string a int
        string cadena = "10";
        int numero = int(cadena)
    \end{lstlisting}
    \section{Sintaxis de superficie}

\end{document}